\documentclass[a4paper,11pt]{article}

\usepackage{amsmath,amssymb,amsthm}

\title{一次元キャプチャ碁 $G_N$ の定義}
\author{}
\date{}

\theoremstyle{definition}
\newtheorem{definition}{定義}

\begin{document}

\maketitle

\begin{definition}[一次元Capture go $G_N$]
自然数 $N$ に対し,以下のルールセット$G_N$とする。

\begin{itemize}
    \item \textbf{盤面:} 

    \[
        V = \{1,2,\dots, N\}
    \]
    を盤面とする。

    \item \textbf{状態:}
    各頂点 $v \in V$ は値
    \[
        s(v) \in \{0, B, W\}
    \]
    を取り,それぞれ空点($0$),黒石($B$),白石($W$)を表す。
    初期状態は
    \[
        s(v)=0 \qquad (v\in V)
    \]
    とする。

    \item \textbf{手番と着手:}
    黒(先手)と白(後手)は交互に手番を持ち,
    \[
        s(v)=0
    \]
    を満たす頂点を 1 つ選び,自色の石を置く。

    \item \textbf{捕獲(勝利条件 1):}
    着手直後,相手色の石が形成する連結成分(連)を $C$ とする。
    その呼吸点(隣接する空点)が
    \[
        \Lambda(C) = \varnothing
    \]
    となった場合,その連は取り除かれ,この状況が発生した手番プレイヤーを勝者とする。
    すなわち,捕獲が生じた瞬間に即時勝利とする。

    \item \textbf{着手不能(勝利条件 2):}
    自殺手(自分の石の連の呼吸点を 0 にする着手)を除き,合法手が存在しない場合,
    手番プレイヤーの敗北とする。
\end{itemize}
\end{definition}

\end{document}
